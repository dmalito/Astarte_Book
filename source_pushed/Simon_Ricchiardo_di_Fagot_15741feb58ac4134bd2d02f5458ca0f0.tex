\section{Simon Ricchiardo di Fagot}\label{simon-ricchiardo-di-fagot}

Tags: NPC Creatore: Lorenzo Ispirazione: Enzo Miccio Luogo: Disharta

\section{\texorpdfstring{\textbf{Simon Ricchiardo di
Fagot}}{Simon Ricchiardo di Fagot}}\label{simon-ricchiardo-di-fagot-1}

\begin{center}\rule{0.5\linewidth}{0.5pt}\end{center}

\begin{figure}
\centering
\includegraphics{02_DUCA.png}
\caption{02\_DUCA.png}
\end{figure}

Informazioni Generali

Età: 40

Data di nascita: 1984

Luogo di nascita: Fagot

Razza: Umano

Classe:

Alleati:

Nemesi:

Alias:

Professione: Duca del Fagot

\begin{center}\rule{0.5\linewidth}{0.5pt}\end{center}

\subsection{1. Descrizione Generale}\label{descrizione-generale}

\begin{center}\rule{0.5\linewidth}{0.5pt}\end{center}

Il Duca Simon Ricchiardo di Fagot è un nobile di nascita, noto per la
sua spocchia e la sua aria aristocratica. La sua presenza è spesso
accompagnata da una sensazione di superiorità, e non si sforza affatto
di nascondere il suo disprezzo per chi considera socialmente inferiore.

\begin{quote}
\emph{``Esteban, raggiungimi nei miei appartamenti tra 10 minuti\ldots{}
E vieni senza camicia!''}
\end{quote}

\subsection{2. Biografia}\label{biografia}

\begin{center}\rule{0.5\linewidth}{0.5pt}\end{center}

Simon è nato in una famiglia nobile con una lunga storia di servizio
all'Impero di Disharta. Cresciuto con una forte educazione
aristocratica, ha ricevuto un addestramento adeguato alla sua posizione
come futuro Duca del Fagot. Sin da giovane, ha dimostrato un impegno per
il benessere del suo ducato e della sua futura posizione. Diverse sono
le voci che suggeriscono che la sua sessualità possa essere al di fuori
degli schemi tradizionali, motivo per cui ha sempre rifiutato proposte
di matrimonio.

\subsection{3. Carriera}\label{carriera}

\begin{center}\rule{0.5\linewidth}{0.5pt}\end{center}

Simon ha iniziato a partecipare agli affari politici dell'Impero fin da
giovane. Ha svolto incarichi diplomatici e ha rappresentato l'Impero in
varie occasioni, guadagnandosi la reputazione di un diplomatico esperto.
La sua nomina come promesso sposo di Leona è stata vista come un
tentativo di consolidare ulteriormente l'unità dell'Impero, in quanto il
ducato del Fagot ha un peso considerevole all'interno degli equilibri
politici dell'Impero.

\subsection{4. Personalità}\label{personalituxe0}

\begin{center}\rule{0.5\linewidth}{0.5pt}\end{center}

La personalità di Simon è notoriamente spocchia e sprezzante. Si
considera superiore agli altri a causa del suo lignaggio nobiliare e
della sua ricchezza. È noto per il suo disprezzo verso i poveri e per il
suo scetticismo nei confronti delle donne. Inizialmente, era destinato a
sposare la principessa, un'unione che avrebbe consolidato il suo potere
e il suo status nella corte imperiale. La fuga di Leona lo ha lasciato
umiliato e disonorato di fronte agli occhi della nobiltà e
dell'imperatore. Simon è considerato dai suoi detrattori come un
individuo arrogante e insensibile, il che lo ha reso impopolare tra
molte fasce della popolazione. La sua presenza nell'Impero di Disharta è
vista da molti come una fonte di instabilità e conflitto, piuttosto che
di stabilità.

\subsection{A. Coinvolgimenti in Eventi
Recenti}\label{a.-coinvolgimenti-in-eventi-recenti}

\begin{center}\rule{0.5\linewidth}{0.5pt}\end{center}

\href{Untitled\%20Database\%20fcbf6eebf0c34c96920278adf0753309.csv}{Untitled
Database}

\subsection{B. Aggiornamenti}\label{b.-aggiornamenti}

\begin{center}\rule{0.5\linewidth}{0.5pt}\end{center}

\href{Untitled\%2085f40f3d6ee74c04a22fd2d8e514fbd8.csv}{}
