\section{Vallontano}\label{vallontano}

Tags: Città, Luogo Creatore: Lorenzo

\section{Vallontano}\label{vallontano-1}

\begin{center}\rule{0.5\linewidth}{0.5pt}\end{center}

\url{drawing-small-village-with-mountain-background_898324-886.avif}

Informazioni Generali

Tipo di Luogo: Villaggio

Dimensioni:

Altitudine: 1600 m slm

Popolazione: 214 (2015)

Paese: Mitegard Meridionale

Luogo: Valtara

Alleata con: Eldrid

Attività: Allevamento, Agricoltura

\begin{center}\rule{0.5\linewidth}{0.5pt}\end{center}

\subsection{1. Descrizione Generale}\label{descrizione-generale}

\begin{center}\rule{0.5\linewidth}{0.5pt}\end{center}

Vallontano è l'ultimo villaggio ancora abitato della Valle dell'Eldrio,
tra le maestose Montagne Astrane, sulle rive del Fiume Eldrio. Questo
fiume scorre anche attraverso la grande città stato di Eldrid, che si
trova a diverse giornate di viaggio a valle. Nonostante la situazione
difficile, Vallontano è un luogo pittoresco, circondato da paesaggi
mozzafiato e natura incontaminata. Le sue case in stile rustico, ora in
rovina, conservano il fascino di un'epoca passata. Tuttavia, la
popolazione è preoccupata per il futuro del loro villaggio e spera in
una rinascita delle loro terre.

\subsection{2. Storia}\label{storia}

\begin{center}\rule{0.5\linewidth}{0.5pt}\end{center}

Vallontano un tempo era un fiorente villaggio agricolo che prosperava
grazie alla fertilità delle terre della Valle dell'Eldrio. Tuttavia, nel
corso degli anni, a causa di varie sfide, tra cui cambiamenti climatici
e attacchi da parte di creature misteriose provenienti dalle Montagne
Astrane, la popolazione del villaggio si è notevolmente ridotta. Ora è
un luogo semi-abbandonato, con case vuote e campi incolti.

Vallontano è stato scelto come primo rifuggio della principessa Leona di
Disharta per la sua fuga per la sua condizione. La principessa sperava
di sfuggire all'attenzione dei suoi persecutori rimanendo nascosta tra i
pochi abitanti del villaggio, ma dovette ripartire quando i protettori
della gilda di Eldrid la trovarono.

\subsection{3. Popolazione}\label{popolazione}

\begin{center}\rule{0.5\linewidth}{0.5pt}\end{center}

La popolazione di Vallontano è estremamente ridotta, come in tutta la
vallata. Attualmente, il villaggio conta poche centinaia di abitanti,
principalmente anziani. La giovane generazione ha lasciato il villaggio
alla ricerca di opportunità in luoghi più grandi e prosperi, come Eldrid
e le floride pianure circostanti.

\subsection{4. Economia}\label{economia}

\begin{center}\rule{0.5\linewidth}{0.5pt}\end{center}

In passato, l'agricoltura era la principale fonte di sussistenza per gli
abitanti di Vallontano. Tuttavia, con la diminuzione della popolazione,
la produzione agricola è stata ridotta al minimo. Alcuni dei pochi
residenti rimasti si dedicano alla pesca nel Fiume Argenteo e nel fiume
Eldrio, cercando di rifornire Eldrid con il loro pescato.

\subsection{5. Cultura}\label{cultura}

\begin{center}\rule{0.5\linewidth}{0.5pt}\end{center}

Vallontano è rimasto un luogo isolato e conservatore. Gli abitanti
mantengono antiche tradizioni agricole e alcuni ricordi delle feste
locali. La gente del villaggio è conosciuta per la loro ospitalità e la
loro accoglienza verso i viaggiatori che occasionalmente passano di lì.
