\section{20 naturale}\label{naturale}

Tags: regola

\textbf{Il 20 naturale in un check/tiro salvezza non significa
automaticamente successo.}

Ci sono diversi motivi per seguire questa regola:

\begin{enumerate}
\def\labelenumi{\arabic{enumi}.}
\tightlist
\item
  \textbf{Conformità alle regole ufficiali:} Questa regola rispetta le
  linee guida ufficiali di D\&D, che non stabiliscono il 20 naturale
  come un successo automatico in ogni situazione.
\item
  \textbf{Equità nel gioco:} L'uso del 20 naturale come successo
  automatico può portare a situazioni paradossali, dove un personaggio
  privo di preparazione o competenze specifiche può riuscire in sfide
  che dovrebbero essere al di là delle sue capacità. Ad esempio, un
  personaggio senza una gamba non dovrebbe essere in grado di scalare un
  muro con un solo colpo, anche con un 20 naturale.
\item
  \textbf{Elemento aleatorio:} Il tiro del dado rappresenta l'elemento
  aleatorio nelle capacità di un personaggio, piuttosto che un ``jolly''
  che consente di superare qualsiasi sfida. Questo elemento di casualità
  aggiunge profondità al gioco e riflette la natura incerta della
  realtà.
\end{enumerate}

Tuttavia, il Master ha sempre l'ultima parola. Se una sfida ha un grado
di difficoltà (CR) di 25, è consigliabile che il tiro del dado più i
modificatori del personaggio raggiungano almeno 25 per avere successo.
Il 20 naturale può comunque essere utilizzato come elemento di gioco di
ruolo aggiuntivo. Ad esempio, se un giocatore supera una sfida con un 20
naturale, il Master può consentire al giocatore di eseguire l'azione in
modo spettacolare o vistoso, aggiungendo un elemento di narrazione e
drammaticità alla scena.

Ricorda che queste regole sono flessibili e possono essere adattate alle
preferenze del gruppo di gioco, ma servono da base per mantenere un
equilibrio nel gioco e una narrativa coerente.
