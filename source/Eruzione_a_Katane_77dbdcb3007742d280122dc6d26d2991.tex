\section{Eruzione a Katane}\label{eruzione-a-katane}

Tag: One-Shot Creature Coinvolte: Flumph, Terramaestra Data Sessione:
September 16, 2023 Giocatori Coinvolti: Branda Gunnarsson, Gnomio
Cartonio, Stalag Might Luoghi: Bellavalle, Katane Master: Luca NPC
coinvolti: Virginio Cotti

\href{Stalag\%20Might\%2082c32b9a749a4543aae2e78474794fe6.md}{Stalag
Might} e
\href{Gnomio\%20Cartonio\%20765b9ff9da1e49eb8401026d43d749a9.md}{Gnomio
Cartonio} rispondono alla chiamata del quartiermastro della sede di
Bellavalle, il dottor \textbf{Virginio Cotti}. Quest'ultimo vuole
organizzare una spedizione scientifica nel vicino villaggio di
\textbf{Katane} per indagare sulla misteriosa e improvvisa comparsa di
quello che sembra un vulcano attivo. Perciò seleziona Stalag e Gnomio
per la loro esperienza in geologia e archeologia. Ai due studiosi
affianca come guardia del corpo
\href{Branda\%20Gunnarsson\%20f8c7d7de2863418e9d8a94419f4b2d74.md}{Branda
Gunnarsson} , barbaro della gilda con notevole esperienza nel cacciare e
combattere le creature più pericolose della regione. Dopo un velocissimo
briefing, i tre partono.

Giunti a Katane, fanno una sosta all'unico emporio del villaggio (che
non è proprio rifornitissimo ma almeno è convenzionato con la gilda)
dove ottengono i tre oggetti magici: \textbf{Gaviscon}, \textbf{Vivin C}
e \textbf{Paracetamolo}.

Mentre passeggiano per le strade di Katane, notano una serie di fenomeni
riconducibili agli effetti di un'eruzione vulcanica: cumuli di cenere ai
bordi delle strade e forte odore di zolfo. Grazie al suo fiuto da
segugio, Stalag rintraccia la fonte dell'odore e il party giunge alle
pendici di una vicina collina dove trova l'ingresso di una caverna. Dopo
un piccolo contrattempo dovuto alla superficialità dei nostri
avventurieri che pensano sia una buona idea accendere un fuoco in una
zona ad alta concentrazione di zolfo, il party si addentra nella caverna
che si rivela essere un passaggio stretto, buio, umido, caldo e
ricoperto da una strana melma scura il quale conduce ad una grande
grotta interna.

Qui trovano una specie di piscina di un denso liquido acido di colore
verde/giallastro, un po' luminoso e in ebollizione. La piscina è
alimentata da una piccola cascata e poi il liquido scorre via attraverso
un canale di scolo.

A terra ci sono varie carcasse di animali in decomposizione. Qui il
party sente dei forti rumori come dei tuoni provenire in lontananza. Ad
un certo punto appare il flumph che è un essere fluttuante a mezz'aria a
forma di medusa che emana una luce fluorescente verdastra. Questo cerca
inizialmente di approcciare pacificamente il party che però non ricambia
con la stessa gentilezza. Spaventato dall'aggressività di Branda, il
flumph tiene le distanze e si fa seguire dal party attraverso un
labirinto di cunicoli stretti e bui.

All'uscita del labirinto Il party con il flumph si ritrovano in un'altra
grande camera semi allagata da grandi pozze di liquido acido.

Qui la temperatura è talmente alta che ci sono getti di vapore tipo
geyser provenienti dalle pareti. A differenza della camera precedente
qui l'acido non scorre, ma sembra stagnante.

Il flumph si offre di fare strada ma, a causa della negligenza del party
che non riesce a proteggerlo, subisce l'attacco mortale di uno Stirge,
che è un mostriciattolo volante con una proboscide in testa. Sembra un
incrocio tra un grosso pipistrello e una zanzara.

Alcuni Stirge si attaccano alle pareti e sembra che succhiano da esse e
a volte attaccano gli avventurieri.

Il party si fa strada alla cieca dovendo superare le pozze di acido più
volte con salti che non sempre finiscono bene, infatti il povero Gnomio
scivola una volta e si ferisce alle gambe.

Arrivati alla fine del percorso ad ostacoli, condito da acrobazie e
improvvisi attacchi degli Stirge, il party trova un piccolo pozzo
contenente dell'acido e al suo interno c'è una piccola fiammella accesa.

Gnomio versa delle gocce del magico Gaviscon nel pozzo e immediatamente
tutte le pozze di acido della caverna diventano di color biancastro e
poi si solidificano diventando una gomma morbida su cui si può camminare
senza problemi.

Il party trova la via per raggiungere la camera successiva.

Si percepiscono delle forti folate di vento freddo a risucchiare verso
l'alto che sono seguite da un forte rumore secco che sembra uno scoppio.

Molto alto è il soffitto dal quale pendono due strutture giganti a forma
di albero capovolto.

Su questi c'è del fluido viscido che gocciola giù. Pozzanghere di questo
fluido si possono trovare qua e là nella camera. Il fluido ricopre le
strutture pendenti alla stessa maniera in cui una grossa ragnatela si
deposita sui rami di un albero, formando una specie di rete. A terra ci
sono anche delle stalagmiti e sul soffitto delle stalattiti. Il party
scopre con grande sorpresa che questi sono in realtà dei Darkmantle e
dei Pierce sopiti, che attaccano perché vedono la loro tana minacciata.

In un angolo della stanza c'è un altro pozzetto come quello della
precedente camera. Questo però contiene dell'acqua. Stalag immerge delle
pietre magiche di Vivin C, che si dissolvono lentamente provocando
l'ebollizione dell'acqua. Immediatamente il fluido viscido scompare e le
strutture a forma di albero riprendono il loro colore originale e
sembrano rinvigorire. A questo punto il party si trova di fronte ad un
bivio, e decide di prendere una delle due strade che porta alla Camera
successiva.

È un ambiente unico molto ampio illuminato da alcuni gruppi di cristalli
splendenti sparsi in giro. Emettono delle piccole scariche
elettrostatiche.

I cristalli splendono ad intermittenza, alcuni più debolmente di altri e
alcuni non splendono affatto ma sembrano come congelati. In alcune zone
il terreno è una lastra di ghiaccio e la temperatura è notevolmente più
bassa in questa camera. Dopo uno scontro con alcune creature che abitano
il luogo, il party trova il terzo pozzetto che contiene del ghiaccio.
Branda spruzza qualche soffio del magico Paracetamolo e il ghiaccio in
tutta la stanza si scioglie immediatamente. Di conseguenza, tutti i
cristalli che prima erano spenti o congelati adesso si sono come
riaccesi e cominciano ad emettere luce.

Tornati al bivio precedente, il party imbocca l'unica via rimasta come
alternativa. Ad un certo punto si ritrovano un una piccola stanza, dove
è tutto buio. Di fronte si trova un muro semicircolare fatto di mattoni
che sembrano ossa. Branda nota un piccolo barlume di luce provenire da
un angolo in alto e ci si arrampica. Tutti si dirigono verso la luce
capendo che si tratta di un cunicolo di uscita. Mentre si trovano nel
cunicolo, una nuova folata di vento forte attraversa il passaggio, ne
segue una esplosione fortissima la cui onda d'urto espelle i nostri eroi
fuori dalla caverna, ritrovandosi all'aperto nella Foresta dei Giganti.

La collina si rivela essere una
\href{Terramaestra\%207baa6c3cd73e483fb4ade190a1dc230e.md}{Terramaestra}.

Grazie alle fantastiche doti di sopravvivenza nella natura di Branda, il
party si orienta e riesce a tornare al villaggio di Katane. Con grande
piacere possono constatare, anche grazie alle felici testimonianze degli
abitanti, che il problema del vulcano attivo sembra essere risolto. Per
sempre?!
