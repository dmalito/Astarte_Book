\section{La fuga della principessa}\label{la-fuga-della-principessa}

Tag: One-Shot Data Sessione: December 30, 2023 Giocatori Coinvolti:
Disis Iulukinfor, Don Tammeo Luoghi: Eldird, Goldendoor, Vallontano
Master: Lorenzo NPC coinvolti: Amarinda Damar, Duca Simon Ricchiardo di
Fagot, Fran, Hakram, Leona Narendra Pahaton, Lucius III di Disharta,
Victor Hamburger

Valtara è in tumulto a seguito della notizia della fuga della
principessa Leona di Disharta, diffusasi ampiamente insieme all'annuncio
dell'imperatore riguardo all'impiego dell'intera gilda dei Protettori
per rintracciarla.

Fran, il quartiermastro della gilda di Eldrid, incarica i Protettori
Salva, Don e Disis (quest'ultimo proveniente da Kos) di recarsi al
Palazzo Reale di Goldendoor, la maestosa capitale dishartiana.
L'obiettivo è condurre un'indagine sotto l'ordine dell'imperatore Lucius
Terzo, che ha richiesto l'esperienza investigativa della gilda.

Partendo a cavallo (con l'eccezione di Don, che opta per la sua fedele
bicicletta), i Protettori raggiungono la capitale in pochi giorni.
Accolti da sontuosi palazzi e strade lastricate d'oro, vengono scortati
al Palazzo Reale e, dopo essere stati annunciati, accedono alla sala del
trono.

L'imperatore Lucius Terzo esorta rapidamente i Protettori a indagare
nelle stanze della figlia, offrendo il supporto della balia Amarinda, la
quale conosce la principessa più di chiunque altro. Mentre i Protettori
lasciano la sala, fa irruzione il duca di Fagot, promesso sposo di
Leona, lamentandosi della situazione e minacciando di far ritorno nei
suoi possedimenti se la principessa non verrà trovata entro la
settimana. Il duca, con la ``r'' moscia e il suo fare effemminato, si
lamenta con l'imperatore per la situazione: si dice offeso e adirato, ma
appare sollevato. Infine il duca si congeda chiedendo a Esteban, un uomo
prestante e muscoloso a suo seguito, di raggiugerlo nelle sue stanze.

Amarinda guida i Protettori nelle stanze di Leona per l'inizio
dell'investigazione. . L'ambiente è ampio, e si divide in 2 stanze, la
camera da letto e il salotto, divise da un ampio bagno.

Mentre Disis esamina le piante senza successo, Don e Salva iniziano a
scrutare attentamente la camera da letto. Salva fa una scoperta
significativa su una scrivania, dove trova delle carte geografiche che
Amarinda riferisce essere il riflesso della passione di Leona per la
geografia, dato che la principessa è confinata nelle mura del palazzo e
non può viaggiare, viaggia con la fantasia. Nel frattempo, Don ispeziona
il comodino accanto al letto, ma non trova molto. Grazie ad un colpo
della balestra di Salva contro il cassetto del comodino (ha il grilletto
facile), questo rivela un doppio fondo. Al suo interno, una sorpresa:
monete d'oro che la principessa Leona sembrava nascondere.

Amarinda, sconcertata dall'apparente necessità di Leona di nascondere
denaro, esprime la sua preoccupazione: ``Leona non dovrebbe neanche
maneggiare il denaro, e poi è tutto a sua disposizione qui a palazza''
spiega la balia.

Nel frattempo, Disis individua una trave del pavimento che sembra
muoversi. Dopo alcuni tentativi infruttuosi, l'intervento muscolare di
Don riesce a sollevare la trave, rivelando un cofanetto segreto. Il
contenuto, una serie di ritratti della principessa, suscita l'interesse
dei Protettori. Amarinda sottolinea che Leona non sa disegnare,
sollevando ulteriori interrogativi sull'origine di quei ritratti.

Salva, indagando tra le coperte del letto, trova delle ciocche di
capelli appartenenti alla principessa. Nel frattempo, Don si dirige
verso una seconda scrivania nella stanza, dove, tra gli altri oggetti,
individua delle poesie dedicate a Leona e firmate con ``H.''. Questa
scoperta porta ad una svolta nell'indagine, alimentando il mistero
intorno alla principessa.

L'attenzione si sposta poi su Disis, il quale ha l'idea geniale di usare
la sua magia per comunicare con le piante della stanza. Dopo
l'incantesimo, le piante rivelano informazioni sorprendenti: la
principessa intratteneva una relazione amoros con un individuo definito
dai fiori come un ``negro''. Le loro frequenti visite alla locanda
dell'orco ubriaco emergono come un altro tassello del puzzle.

Con le rivelazioni che emergono, la realtà della situazione inizia a
delinearsi per i Protettori e per Amarinda, che non trattiene le
lacrime. Determinati a proseguire nell'indagine, i Protettori continuano
la loro ricerca. Trovano ulteriori poesie, un sottobicchiere proveniente
dalla locanda menzionata e un pezzo di stoffa legato alle grate della
finestra, segno inequivocabile di una possibile fuga.

Decisi a scoprire ulteriori dettagli, i Protettori si preparano a
dirigere le loro attenzioni verso la locanda in questione. Prima di
partire, Don decide di esibirsi al pianoforte nella stanza, suonando una
canzone cara a lui e conosciuta come l'inno nazionale dishartiano
``Visetto Scuro''.

Scortati da guardie, i Protettori raggiungono la locanda, ancora chiusa.
Bussano alla porta e vengono accolti da Victor, l'imponente orco
proprietario. Victor, riconoscendo i Protettori, si mostra inizialmente
restio a cooperare.

Infine Victor, intimidito dai protettori, sicuri della connessione tra
la locanda e lafuga della principessa, confessa il suo coinvolgimento
attaccandoli.

Segue uno scontro con i cuochi armati di Victor, durante il quale i
Protettori dimostrano la loro maestria nel combattimento. La situazione
si risolve con la resa di Victor, disperato per aver messo a rischio la
vita dei suoi dipendenti. Victor chiede ai Protettori di arrestarlo, ma
tace riguardo ai dettagli del suo coinvolgimento con la fuga della
principessa.

Don, mostrando la sua compassione, decide di curare uno dei cuochi
feriti. Questo atto tocca profondamente Victor, che ringrazia e, a
malincuore, rivela che Leona è ormai lontana, fuggita in un luogo
sicuro. Le informazioni aggiuntive ottenute da Disis, grazie alla sua
magia, non contribuiscono a risolvere completamente il mistero.

I Protettori decidono di non arrestare Victor, lasciandolo in pace, ma
intraprendono ulteriori indagini all'interno della locanda. Attraverso
l'aiuto di un'altra pianta chiacchierona (che incantesimo utile!),
trovano un passaggio segreto che conduce a una stanza da letto. È da qui
che Leona e il suo amato sono fuggiti.

Ulteriori indagini portano alla luce atlanti geografici e un'altra
poesia, fornendo finalmente indicazioni concrete sulla destinazione
della principessa: Vallontano, un piccolo paesino nella valle
dell'Eldrio.

I Protettori tornano al Palazzo Reale e riferiscono le loro scoperte. Il
giorno successivo, partono per Vallontano, un pittoresco villaggio
rurale, dove cercano di ottenere informazioni interrogando i residenti.
La loro ricerca si svolge con un'irruenza non professionale da parte di
Salva, che spaventa i passanti con la sua balestra. Tuttavia, questa
tattica si rivela efficace, conducendo i Protettori alla rivelazione
della posizione dei fuggitivi grazie a una signora spaventata.

Indirizzati dalla signora, i Protettori raggiungono la casa in cui si
crede che i fuggitivi si nascondano. Tuttavia avvicinandosi alla casa
vedono due persone fuggire via.

I protettori decidono di investigare all'interno dell'abitazione, ma un
incendio improvviso la avvolge, rendendo evidente che qualcuno ha
intenzionalmente lanciato dardi infuocati. I Protettori escono
precipitosamente, incapaci di localizzare i fuggitivi, ma convinti che
siano stati loro a provocare l'incendio.

Disis, grazie alla sua magia, riesce a individuare la posizione esatta
della principessa. Lanciando l'incantesimo, i Protettori vengono
teletrasportati a poche centinaia di metri dalla casa in fiamme, dove la
principessa si nascondeva dietro ad un grosso albero. Quando Don si
avvicina a Leona, apparentemente spaventata, l'amante Hakram emerge dal
nulla, grazie a un mantello dell'invisibilità.

Hakram minaccia i Protettori, chiedendo di essere lasciato libero o
promettendo loro la morte per mano sua. Don cerca di persuaderlo, ma
Hakram è deciso all'azione. Tuttavia, prima che possa scatenare il suo
attacco, Leona interviene, rivelando la sua gravidanza da Hakram e
chiedendo pietà.

Mosso a compassione, il gruppo di Protettori decide di acconsentire alla
loro richiesta di libertà. Don, tuttavia, prevede le conseguenze
dell'indagine e chiede a Leona un oggetto personale. La principessa gli
consegna un ciondolo con l'immagine della madre defunta.

Ritornati a Goldendoor, i Protettori riferiscono all'imperatore una
versione distorta della verità. Affermano che i fuggitivi sono periti in
mare, e il ciondolo è l'unico oggetto ritrovato della principessa.
L'imperatore, furioso non tanto per la presunta morte della figlia ma
per il colpo al suo potere, li caccia dal palazzo, gettando Disharta in
un futuro incerto.
