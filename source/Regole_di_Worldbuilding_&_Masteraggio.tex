\section{Regole di Worldbuilding \&
Masteraggio}\label{regole-di-worldbuilding-masteraggio}

\begin{enumerate}
\def\labelenumi{\arabic{enumi}.}
\tightlist
\item
  \textbf{REGOLE DI WORLDBUILDING}

  \begin{enumerate}
  \def\labelenumii{\arabic{enumii}.}
  \tightlist
  \item
    \textbf{Ambientazione Temporale}: È essenziale che tutte le
    avventure si svolgano esclusivamente nel presente.
  \item
    \textbf{Geopolitica e Conflitti}: La creazione di storie che
    potrebbero avere un impatto significativo sulla geopolitica
    regionale o mondiale è strettamente vietata, a meno che non sia
    stata previamente discussa e approvata dai Capi Master. Tuttavia, è
    permesso creare rivalità tra città o conflitti tra piccole fazioni,
    previa consultazione con i Capi Master.
  \item
    \textbf{Esseri Divini e Creature Mitologiche}: L'inclusione di
    divinità creatrici, esseri divini o creature mitologiche (come i
    draghi) è consentita solo previa consultazione e approvazione dei
    Capi Master.
  \item
    \textbf{Creazione di Creature}: Evitare l'eccessiva creazione di
    nuove creature o mostri. Ogni creazione deve essere coerente con il
    mondo di gioco esistente.
  \end{enumerate}
\item
  \textbf{REGOLE DI MASTERAGGIO}

  \begin{enumerate}
  \def\labelenumii{\arabic{enumii}.}
  \tightlist
  \item
    \textbf{Affiliazione dei Personaggi}: Ogni missione deve avere
    inizio in una sede della Gilda o in un luogo che abbia una chiara
    connessione con essa. I personaggi coinvolti devono essere affiliati
    alla sede della Gilda da cui parte l'avventura, o se ciò non è il
    caso, devono essere fornite giustificazioni adeguate.
  \item
    \textbf{Esperienza e Livellamento}: Alla fine di ogni missione, il
    Master deve assegnare una quantità specifica di esperienza ai
    giocatori. Non è consentito far avanzare direttamente al livello
    successivo; è obbligatorio specificare quanti punti esperienza sono
    stati guadagnati.
  \item
    \textbf{Resoconto delle Sessioni}: Alla fine di ogni sessione, il
    Master deve compilare un resoconto dettagliato degli eventi. Deve
    includere descrizioni approfondite dei personaggi non giocanti
    incontrati, dei luoghi visitati e degli oggetti ottenuti dai
    giocatori. Inoltre, il Master deve fornire almeno due immagini per
    ciascuna descrizione per arricchire l'esperienza di gioco.
  \end{enumerate}
\end{enumerate}
