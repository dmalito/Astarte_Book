\section{Il neonato incasinato}\label{il-neonato-incasinato}

Tag: Dal Vivo, One-Shot Creature Coinvolte: Naskirophis Data Sessione:
August 16, 2023 Giocatori Coinvolti: Disis Iulukinfor, Dorian Be, Pippo
Francfrog, Sabaku no Darude Luoghi: Kos, Pandosia Master: Lorenzo NPC
coinvolti: Arvinio Scarnati, Marpalo Rem, Nicolos, Silvio Berlusconi,
Verde Windrider

\href{Marpalo\%20Rem\%202582afbe3ba04b1ea86091e6d1f4e6ca.md}{Marpalo
Rem} , il quartier mastro della
\href{Gilda\%20dei\%20protettori\%20della\%20Sila\%20Devoti\%20a\%20San\%20Franc\%20e29bb7909af24fee931336355db913d4.md}{\textbf{Gilda
dei protettori della Sila Devoti a San Francesco e ai Lupi}} di Kos, ha
richiesto l'aiuto degli avventurieri dopo che un esploratore giunto
nella sede della Gilda è apparso agitato e terrorizzato. L'esploratore
ha raccontato di aver esplorato un tempio in rovina situato sulle
pendici del Monte Cucuzzo con il suo gruppo di esploratori. Tuttavia,
durante il racconto, ha mostrato segni di crescente delirio che sono
culminati in un attacco incontrollato contro coloro che lo ascoltavano.
L'esploratore viene trattenuto nelle celle della Gilda, ma si toglie la
vita alcuni giorni dopo. Lo stesso giorno giunge una richiesta di aiuto
dal Questore di
\href{Pandosia\%2028129d9d5ac7448d98387dc4262c4704.md}{Pandosia} . In
risposta, Marpalo Rem ha selezionato con cura un gruppo di avventurieri
composto da
\href{Pippo\%20Francfrog\%204d15378e582d4f1db815d957fe064245.md}{Pippo
Francfrog} ,
\href{Dorian\%20Be\%20af030367f8054333912b2dca0de16d6f.md}{Dorian Be},
\href{Disis\%20Iulukinfor\%20e7699726707a41be926c823d67941f78.md}{Disis
Iulukinfor} e
\href{Sabaku\%20no\%20Darude\%209c414f3e551144f4acee665cab478336.md}{Sabaku
no Darude} per indagare sull'attacco subito dalla città a causa di una
bestia demoniaca. Marpalo informa il party che popolazione di Pandosia
sostiene che questa creatura sia un emissario del temibile demone
\href{Naskirophis\%20120e02c652b84f2abeac36fef59c28f6.md}{Naskirophis} ,
demone risvegliato (sempre secondo le voci popolari) da degli
``esploratori ficcanaso''. L'interconnessione tra il folle esploratore e
l'attacco che ha colpito Pandosia si fa più chiara. Dopo essere arrivati
a Pandosia, il gruppo ha trovato alloggio alla locanda ``Le Quercie''.
Qui, la serata prende vita con l'entusiasmo del gruppo, che anima
l'atmosfera in modo vivace e talvolta chiassoso. Tuttavia, ciò crea un
disagio per una famiglia di nani presenti nella locanda. Dorian trova
svago nel fare acrobazie con i membri della famiglia, mentre gli altri
componenti del gruppo intrattengono con suoni e canti. La reazione
dell'oste è di impotenza, sopraffatto dalla stravaganza del gruppo, si
ritrova a pazientare in attesa della fine di questa cacofonia. Nel cuore
della notte, all'interno della locanda, Disis è tormentato da un sogno
inquietante. Nonostante non abbia mai visto il demone prima d'ora, il
suo subconscio lo riconosce immediatamente come il protagonista di
questo incubo. La mattina seguente, Sabaku chiede nella bottega di un
falegname le indicazioni per raggiungere la Questura della città, per
poi guidare il resto del gruppo verso la meta comune. Durante il
tragitto il gruppo si imbatte nel Tempio di
\href{San\%20Nikolos\%20Dibar\%2025d750713dba4816a1b6771821ab3187.md}{San
Nikolos Dibar} , e decide di entrare. Qui, Sabaku ha un breve scambio di
parole con un sacerdote di nome
\href{Nicolos\%20fac31cf73d3d4b70b068ab976e2129e1.md}{Nicolos} . Da lui
ottiene informazioni aggiuntive sulla bestia demoniaca che ha assalito
la città, che viene descritta come un serpente piumato dalle strane
abitudini starnazzanti. Dopo l'incontro con il sacerdote, il gruppo
raggiunge la Questura, dove stabilisce un incontro con il Questore
\href{Arvinio\%20Scarnati\%209ff14cda64684ea88c6772adc5b63f01.md}{Arvinio
Scarnati}. Il Questore accoglie gli avventurieri immerso nella sua
piscina, mostrando fin da subito un comportamento snob e presuntuoso. Il
suo ufficio sembra più una spa che un luogo di lavoro, con un
arredamento barocco che suggerisce un certo gusto per l'eleganza e il
lusso. Egli dimostra una certa scetticità nei confronti delle voci in
circolazione riguardo all'attacco che ha colpito la città. Egli inclina
a credere che l'assalto sia stato perpetrato da una feroce bestia
piuttosto che da un demone sovrannaturale. Nonostante la discrepanza
nelle percezioni, il questore propone un'affare ai valorosi
avventurieri: offrire loro un compenso di cento monete d'oro ciascuno al
fine di risolvere tempestivamente la situazione. La loro missione:
individuare un'imponente creatura che possa fungere da capro espiatorio,
eliminare qualsiasi minaccia residua e poi presentare alla cittadinanza
le prove che rassicurino sulla cessazione del pericolo. Gli etici
principi dei Protettori avrebbero vietato categoricamente l'accettazione
di una simile richiesta. La loro Gilda condanna severamente ogni forma
di corruzione, stabilendo come sanzione l'espulsione immediata dal suo
contesto. Tuttavia, in uno strano e controverso capovolgimento, gli
avventurieri accettano comunque l'offerta e Pippo Francfrog, senza farsi
vedere, trova il tempo per rilassarsi nella piscina (ma senza farsi
vedere). Sotto l'occhio vigile di due guardie, gli avventurieri vengono
scortati fino al tempio abbandonato. Qui, lasciati a loro stessi,
affrontano l'oscura sfida che si staglia innanzi. Gli eroi si
avventurano all'interno del tempio, trovandolo inaspettatamente occupato
da un gruppo di accoliti appartenenti alla sinistra
\href{Setta\%20del\%20Sangue\%202859c4de945546eda0cee6fb151ef956.md}{Setta
del Sangue} . Questi individui si concentrano nel tentativo di aprire un
portale magico, ignari dell'arrivo dei nuovi visitatori. Tuttavia, i
passi degli avventurieri disturbano la loro cerimonia, attirando
l'attenzione degli accoliti che si trasformano quasi istantaneamente in
una sinistra entità demoniaca, un ibrido mostruoso con cinque teste
dalla feroce espressione. È probabile che questo sia lo stesso demone
responsabile dell'attacco a Pandosia, sebbene il movente rimanga avvolto
nel mistero. La battaglia è imminente e l'aria è intrisa di tensione
mentre gli accoliti-demoni attaccano gli avventurieri con foga. La
ferocia del conflitto culmina nell'unione delle loro forme, dando
origine a una terribile Papera dalle cinque teste, la
\href{Paperidra\%20fb14298e444f4526a49b518790283588.md}{Paperidra} .
Nonostante la difficoltà, e gli attacchi fecali della bestia demoniaca,
gli avventurieri dimostrano il loro coraggio e la loro abilità,
riuscendo a sconfiggere questo possente nemico. L'atmosfera si calma una
volta che la minaccia è debellata, rivelando agli occhi degli
avventurieri un quadro agghiacciante. Tra le macabre reliquie rituali
giace un neonato, evidentemente destinato al sacrificio in favore del
demone. La tragica scoperta agisce come una rivelazione inquietante.
Nelle tasche dei vinti, gli avventurieri trovano delle monete dal valore
singolare. La comprensione giunge subito: queste monete sono la chiave
per aprire il portale magico. L'atto di gettarne una in una vasca al
centro del tempio scatena una risposta magica: l'acqua all'interno della
vasca si anima di una luminosa luce blu fosforescente. Il gruppo decide
di abbracciare questa enigmatica opportunità. Senza esitazione, si
immergono nell'acqua luminosa e, in un istante, vengono risucchiati dal
portale. Con loro, portano il prezioso fardello dell'infante,
garantendone la salvezza nel mondo sconosciuto che ora li attende. Il
gruppo si trova ad affrontare la sfida di attraversare tre piani
dimensionali, ciascuno apparentemente identico agli altri, ma ognuno con
prove uniche. Per accedere a ciascun livello, il gruppo impiega una
moneta per riattivare il portale. Sebbene esistano parole d'ordine per
attraversare i piani senza sfide, il gruppo non le conosce, e deve
affrontare ogni pericolo. Nel primo piano, i membri si ritrovano a
scontrarsi con spettri inquieti. Nel contesto del campo di battaglia,
gli eroi vengono teletrasportati continuamente in una posizione
differente, dove un nuovo spettro è pronto a tormentarli. Questi spettri
rivelano agli avventurieri scene drammatiche del passato. Sabaku è
afflitto da visioni inquietanti: vede i suoi genitori sciogliersi in
sabbia nel deserto in cui è nato, un'immagine che scuote profondamente
le sue emozioni. In seguito, è tormentato dalla visione della compagna
di una vita, che gli strappa il cuore. D'altra parte, Pippo Francforg è
tormentato da spettri del passato. I suoi compagni di sempre lo accusano
di aver abbandonato l'arte e lo dipingono come un traditore,
costringendolo a confrontarsi con il peso delle sue scelte passate.
Disis invece riesce a resistere alle visioni dei fantasmi, mentre
Dorian, che ha in custodia il bambino, non viene neanche toccato.
Curiosamente, i fantasmi sembrano ignorare completamente il bambino che
Dorian protegge, quasi come se fossero intimoriti dalla sua presenza.
Grazie a questa reazione singolare del bambino, il gruppo riesce a
superare l'ostacolo dei fantasmi. Il secondo piano presenta una sfida
ben diversa, mettendo il gruppo di fronte a un massiccio
\href{Colosso\%20di\%20Pietra\%20e86bfcf7509c43f68ad91526717e23b0.md}{Colosso
di Pietra} . Questo golem è manovrato da una creatura che ne controlla
ogni movimento, protetta da un impenetrabile campo di forza. La potenza
distruttiva del golem mette gli avventurieri sotto pressione, portandoli
al limite. In un momento critico, Sabaku, con il bambino al seguito,
rischia di essere colpito e ucciso dal poderoso fendente della spada del
golem. Tuttavia, il bambino compie un gesto straordinario, deviando
l'attacco e infrangendo la spada del golem. È evidente che il bambino
possiede abilità sovrannaturali. L'uso di tali abilità fa cadere il
pargolo in un sonno profondo, che lo accompagnerà per tutto il resto
dell'avventura. Con grande sforzo, il gruppo riesce a disattivare il
campo di forza del burattinaio e ad eliminarlo, rendendo inoffensivo il
golem. Nel terzo e ultimo livello, il gruppo affronta direttamente
Naskirophis, il demone temuto. Il demone emerge rompendo le pareti del
tempio: il suo corpo allungato sembra un cordone ombelicale, unto di
sangue e guano. La testa finale del demone ricorda vagamente quella di
un neonato, ma sulla fronte si apre una bocca infestata da denti
acuminati. Naskirophis esige nutrimento e propone un patto, offrendo al
gruppo un potere illimitato in cambio del bambino che essi proteggono.
La risposta negativa è pronta e decisa, costringendo il demone a passare
al combattimento. Una scena drammatica si dipana: Sabaku è posseduto dal
demone stesso, costretto a consegnare il neonato nelle grinfie
dell'entità malvagia. Ma proprio quando sembra che la situazione sia
senza speranza, l'audace Dorian interviene con feroci pugni, cercando di
tenere a bada il demone e difendere il bambino. La sua audacia lo porta
addirittura a finire nelle fauci del demone, dimostrando di essere
pronto anche a sacrificare sé stesso per la salvezza del piccolo. In un
momento cruciale, il fuoco dell'ardore e della determinazione brilla nei
cuori di Disis e Pippo. Con potenti attacchi infuocati, si scagliano
contro il demone con coraggio e abilità. La loro feroce azione si
dimostra cruciale, costringendo il demone a ritirarsi e proteggendo il
neonato dalla sua fame insaziabile. Nonostante i suoi poteri psichici,
il demone è impotente di fronte alla determinazione del gruppo, pronto a
cancellare persino il ricordo della sua esistenza. In punto di morte, il
demone emette un ultimo respiro, col quale pianta ``un seme''
all'interno del bambino, lasciando un'enigmatica traccia del suo
passaggio. L'acqua inizia a defluire dalla vasca, si espande
nell'ambiente e improvvisamente trasporta tutto il gruppo nella prima
sala del tempio.

Il party è esausto e provato, avendo affrontato una sfida titanica per
sconfiggere il demone. Molti membri presentano ferite profonde e serie.
Tuttavia, il loro sforzo non passa inosservato. In questo momento di
difficoltà, il loro salvataggio arriva sotto forma di Verde Windrider e
un distinto gruppo di paladini. Questi guardiani della giustizia li
conducono in salvo e rivelano la vera natura del demone: Naskirophis da
sempre utilizza gli accoliti della setta del Sangue come burattini per i
suoi malvagi intenti. In cambio di neonati (di cui il demone si cibava)
Naskirophis donava poteri sovraumani ai suoi accoliti. Fu proprio grazie
a questi poteri che la Setta del Sangue si infiltrò nell'Ordine dei
Paladini di San Francesco, provocando la guerra del Sangue. Verde
Windrider, il capo della Gilda di Kos, spiega cosa è successo agli
avventurieri: quando l'esploratore ha aggredito i suoi interlocutori, è
stato lui stesso a decidere di confinare l'esploratore nelle celle della
Gilda, almeno fino a quando non si fosse ripreso. Durante questa
conversazione, Windrider comprese che gli esploratori avevano
involontariamente liberato il malvagio demone noto come Naskirophis,
precedentemente imprigionato dagli stessi Paladini di San Francesco
durante la Guerra del Sangue, guerra combattuta contro la Setta del
Sangue. Verde si è preparato per un viaggio al Santuario di San
Francesco per consultarsi con gli altri Paladini, incluso suo padre
\href{Sam\%20Windrider\%20568502cb37144ff8990e673a9cd67375.md}{Sam
Windrider} , per stabilire una strategia di intervento. La situazione
richiedeva una riflessione attenta, poiché il ritorno di Naskirophis
minacciava la pace che era stata raggiunta dopo la Guerra del
Sangue.Tuttavia, gli eventi hanno preso una svolta inaspettata con
l'arrivo di una richiesta di aiuto dal Questore di Pandosia e la notizia
del suicidio dell'esploratore. Marpalo Rem, quartier mastro della Gilda
di Kos, ha agito rapidamente, inviando gli avventurieri a Pandosia senza
aspettare ulteriori deliberazioni con i Paladini. In riconoscimento del
loro coraggio e sacrificio, i membri del party vengono promossi a
capitani. Tuttavia, il bambino porta con sé il seme di Naskirophis,
un'eredità di potere oscura destinata a un futuro incerto. Si rivela che
il bambino è predestinato a diventare la seconda reincarnazione del
demone. Nonostante questa minaccia imminente, i membri del gruppo
decidono di intraprendere una strada diversa. Sono risoluti a non
uccidere il bambino, ma piuttosto a affidarlo alla custodia dei
paladini. Questi uomini e donne retti lo cresceranno come un futuro
paladino all'interno del santuario di San Francesco. L'obiettivo è di
educarlo seguendo i nobili valori di San Francesco, sperando di piegare
il suo destino lontano dalla tragedia che lo attende. Prima di
congedarsi dal bambino, il gruppo prende una decisione importante. Gli
danno un nome, un nome di grande rilevanza che incarichi il bambino di
portare avanti con onore l'eredità dell'Ordine dei Paladini. Con questo
atto, il bambino è ribattezzato
\href{Silvio\%20Berlusconi\%2051c2a61cf7214d26a74744d6ab46e241.md}{Silvio
Berlusconi} , in un augurio di crescita onorabile e nella speranza di un
destino diverso. Con la promessa di un futuro migliore, il gruppo si
separa dal bambino e da tutto ciò che rappresenta. Con gli occhi fissi
verso l'orizzonte, si allontanano, sospinti dal vento dell'incertezza e
dell'opportunità.

\url{https://www.notion.so}
